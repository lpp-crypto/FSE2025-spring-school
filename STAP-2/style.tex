% Time-stamp: <2025-03-06 18:59:29>


%!SECTION! Packages needed 

\usepackage{etex} % use when too many packages, vpv



\usepackage[english]{babel}

\usepackage{cmap}
\usepackage{graphicx}
\usepackage{color,colortbl,spot}
\usepackage{pgfplots}


% \usepackage{utopia}
% Possible fonts:
% mathptmx, helvet, bookman, chancery, charter, libertine, mathptm,
% newcent, palatino, pifont, utopia
% \usepackage[libertine]{newtxmath}
% \usepackage[scaled=0.96]{zi4}


\usepackage{hyperref}
\usepackage{appendixnumberbeamer}

\usepackage{booktabs} % tables, vpv
\usepackage{subcaption}
\usepackage{comment}


\usepackage{amsmath}
\usepackage{amssymb}


\usepackage{multicol}
\usepackage{multirow}
\usepackage{array} % To use monospace font in arrays easily
\newcolumntype{t}{>{\tt}c}



\usepackage[absolute,overlay]{textpos}
\usepackage{overpic}
\usepackage{tikz}
\usetikzlibrary{decorations.pathreplacing}
\usetikzlibrary{arrows}
\usetikzlibrary{arrows.meta}
\usetikzlibrary{patterns}
\usetikzlibrary{shapes}
\usetikzlibrary{shadows}
\usetikzlibrary{calc}
\usetikzlibrary{math}
\tikzstyle{fun}=[draw,very thick,fill=white]
\tikzstyle{fork}=[shape=circle,inner sep=0pt,minimum size=3pt,fill=black]
\tikzstyle{op}=[inner sep=2pt]
\tikzset{every picture/.style={line width=1.2pt}}



% !SECTION! Theme

% \usefonttheme{professionalfonts}

\usepackage{mathspec}
\setmainfont{Inria Serif}[Scale=MatchLowercase]
\setsansfont{Inria Sans}[Scale=MatchLowercase]
\setmathfont(Digits,Latin,Greek,Symbols)[Scale=MatchLowercase]{Inria Serif}
\setmathrm[Scale=MatchLowercase]{Inria Serif}


%!SUBSECTION! Custom colors 
\definecolor{light-gray}{RGB}{240, 240, 240}
\definecolor{dark-gray}{gray}{0.45}
\definecolor{navy}{RGB}{0, 40, 150}
\definecolor{darkgreen}{RGB}{10, 150, 60}
\definecolor{brown}{RGB}{120, 50, 10}
\definecolor{darkred}{rgb}{0.8,0,0}
\definecolor{mypink}{RGB}{200, 100, 100}



%% Inner color theme: lily orchid rose
%% Outer color theme: whale seahorse dolphin
%% Full  color theme: albatross beaver beetle crane dove fly seagull wolverine
% \usecolortheme{beaver}
% \usecolortheme{orchid}


%!SUBSECTION! Completely custom color theme 

\definecolor{mycol1}{RGB}{231, 51, 18}
\definecolor{mycol2}{RGB}{56,66,87}

\definecolor{mybrown}{RGB}{170,105, 57}
\definecolor{mydarkbrown}{RGB}{ 85, 36,  0}

\definecolor{red}{RGB}{192,0,0}
\definecolor{blue}{RGB}{ 50, 60,200}
\definecolor{teal}{RGB}{170,105, 57}

\setbeamercolor{section in toc}{fg=black,bg=white}
\setbeamercolor{alerted text}{fg=red}
\setbeamercolor*{palette primary}{fg=mycol2,bg=white}
\setbeamercolor*{palette secondary}{fg=mycol1,bg=mycol2}
\setbeamercolor*{palette tertiary}{bg=mycol2,fg=mycol1}
\setbeamercolor*{palette quaternary}{fg=white,bg=mycol2} % smoothbars

\setbeamercolor*{sidebar}{fg=darkred,bg=gray!15!white}

% \setbeamercolor*{palette sidebar primary}{fg=darkred!10!black}
% \setbeamercolor*{palette sidebar secondary}{fg=white}
% \setbeamercolor*{palette sidebar tertiary}{fg=darkred!50!black}
% \setbeamercolor*{palette sidebar quaternary}{fg=gray!10!white}


\setbeamercolor{titlelike}{fg=mycol1,bg=gray!15!white}
\setbeamercolor{frametitle}{fg=mycol1,bg=gray!15!white}
\setbeamercolor{frametitle right}{bg=mycol2!60!white}

\setbeamercolor{item}{fg=mycol1}
\setbeamercolor{subitem}{fg=mycol1}
\setbeamercolor{subsubitem}{fg=mycol1}

\setbeamercolor*{separation line}{}
\setbeamercolor*{fine separation line}{}

\setbeamercolor{block title}{fg=mycol1!80!black,bg=mycol1!10!white}
\setbeamercolor{block body}{bg=mycol1!2!white}
\setbeamercolor{block title example}{fg=mycol2!80!black,bg=mycol2!10!white}
\setbeamercolor{block body example}{bg=mycol2!5!white}

%!SUBSECTION! Setting footer 

\setbeamertemplate{footline}[frame number]
% \setbeamertemplate{footline}
% {
% \leavevmode%
% \hbox{%
% \begin{beamercolorbox}[wd=.3\paperwidth,ht=2.25ex,dp=1ex,center]{author in head/foot}%
%   \usebeamerfont{author in head/foot}\insertshortauthor
% \end{beamercolorbox}%
% \begin{beamercolorbox}[wd=.7\paperwidth,ht=2.25ex,dp=1ex,center]{title in head/foot}%
%   \usebeamerfont{title in head/foot}\insertshorttitle\hspace*{3em}
%   \hfill
%   \insertframenumber{} / \inserttotalframenumber\hspace*{1ex}
% \end{beamercolorbox}}%
% \vskip0pt%
% }
\beamertemplatenavigationsymbolsempty




%!SECTION! Abbreviations

\newcommand\smallcite[1]{{\color{gray} \footnotesize \cite{#1}}}


%!SECTION! Custom frame types

% Adding the plan at the beginning of each section starting from now
\newcommand\tocStartsAppearingHere{ %
  \begin{frame}[noframenumbering]
    \frametitle{Outline}
    \tableofcontents[
    sectionstyle=show,
    subsectionstyle=hide,
    subsubsectionstyle=hide] 
  \end{frame}

  \AtBeginSection[] { %
    % \begin{frame}[noframenumbering]
    %   \frametitle{Outline}
    %   \tableofcontents[
    %   currentsection,
    %   sectionstyle=show/shaded,
    %   subsectionstyle=hide,
    %   subsubsectionstyle=hide] 
    % \end{frame}

    \begin{frame}[noframenumbering]
      \frametitle{Plan of this Section}
      \tableofcontents[
      currentsection,
      currentsubsection,
      sectionstyle=show/shaded,
      subsectionstyle=show/hide/hide,
      subsubsectionstyle=hide]
    \end{frame}
  }

  \AtBeginSubsection[] { %
    \begin{frame}[noframenumbering]
      \frametitle{Plan of this Section}
      \tableofcontents[
      currentsection,
      currentsubsection,
      sectionstyle=show/shaded,
      subsectionstyle=show/shaded/hide,
      subsubsectionstyle=hide]
    \end{frame}
  }
}


\newcommand\myTitlePage{
  \frame[plain]{\titlepage \addtocounter{framenumber}{-1}}
}


%!SECTION! Useful macros
%=======================

% source: https://tex.stackexchange.com/questions/28704/defining-a-newcommand-with-variable-name-inside-another-newcommand
% usage: \defcolvar{f}{\mathcal{F}}{blue} creates new macro, \myf,
% which expands into {\color{blue}\mathcal{F}}
\newcommand{\defcolvar}[3]{%
  \expandafter\newcommand\csname my#1\endcsname{{\color{#3} #2 }} %
}

\newcommand\cols[4]{%
  \begin{columns}
    \begin{column}{#1\textwidth}
      \begin{center}
        #3
      \end{center}
    \end{column}
    \begin{column}{#2\textwidth}
      \begin{center}
        #4
      \end{center}
    \end{column}
  \end{columns}
}

\newcommand\showThenHighlight[1]{\only<1>{ #1 }
  \only<2>{\alert{\bfseries #1 }}}



%!SUBSECTION! Math
 
\newcommand\ftwo{\mathbb{F}_2}
\newcommand\proba{\mathsf{Pr}}

\newcommand\F{\mathbb{F}}
\newcommand\Zmod[1]{\mathbb{Z}/ #1 \mathbb{Z}}
\newcommand\field[1]{\mathbb{F}_{2^{#1}}}

\newcommand\openbutterfly[3]{\mathsf{H}^{#1}_{#2, #3}}
\newcommand\newbutterfly[1]{\mathsf{N}_{#1}}
\newcommand\linearspan[1]{\langle #1 \rangle}
\newcommand\rank{\mathsf{rank}}
\newcommand\extract[1]{\mathcal{X}_{#1}}
\newcommand\walshZeroes[1]{\mathcal{Z}_{#1}}
\newcommand\ddtZeroes[1]{\mathcal{Z}^D_{#1}}

\newcommand\lat[1]{\mathcal{W}_{#1}}
\newcommand\ddt[1]{\mathcal{D}_{#1}}
\newcommand\scalarprod[2]{#1 \cdot #2}

\newcommand\spaceInput{\mathcal{V}}
\newcommand\spaceOutput{\mathcal{V}^{\perp}}
\newcommand\allSpaces[1]{\mathcal{S}_{#1}}
\newcommand\admissible[1]{\mathcal{A}_{#1}}
\newcommand\invariant[1]{\mathcal{J}_{#1}}
\newcommand\codebook[1]{\Gamma_{#1}}
\newcommand\eaMappings{\mathcal{M}^{\textrm{EA}}}
\newcommand\nSpaces[1]{s_{#1}}
\newcommand\msb[1]{\mathsf{MSB}(#1)}

\newcommand\matSwap[1]{M_{#1}}
\newcommand\maj{\textrm{maj}}
\newcommand\twistable[1]{#1-twistable}
\newcommand\twotwoMat[4]{\left[ \begin{array}{cc} #1 & #2 \\ #3 & #4 \end{array} \right]}

\newcommand\tklog{TKlog}
\newcommand\tklogmath[1]{\mathcal{TK}_{#1}}

\newcommand\SBox{{\color{darkgreen}S}}
\newcommand\trace[1]{\textrm{Tr}\left( #1 \right)}
\newcommand\conference[1]{{\color{gray}(#1)}}

\newcommand\anomaly{\mathcal{A}}
\newcommand\symField[1]{\mathfrak{S}_{2^{#1}}}
\newcommand\tu[1]{\mathsf{TU}_{#1}}
